% Options for packages loaded elsewhere
\PassOptionsToPackage{unicode}{hyperref}
\PassOptionsToPackage{hyphens}{url}
\documentclass[
]{article}
\usepackage{xcolor}
\usepackage[margin=1in]{geometry}
\usepackage{amsmath,amssymb}
\setcounter{secnumdepth}{5}
\usepackage{iftex}
\ifPDFTeX
  \usepackage[T1]{fontenc}
  \usepackage[utf8]{inputenc}
  \usepackage{textcomp} % provide euro and other symbols
\else % if luatex or xetex
  \usepackage{unicode-math} % this also loads fontspec
  \defaultfontfeatures{Scale=MatchLowercase}
  \defaultfontfeatures[\rmfamily]{Ligatures=TeX,Scale=1}
\fi
\usepackage{lmodern}
\ifPDFTeX\else
  % xetex/luatex font selection
\fi
% Use upquote if available, for straight quotes in verbatim environments
\IfFileExists{upquote.sty}{\usepackage{upquote}}{}
\IfFileExists{microtype.sty}{% use microtype if available
  \usepackage[]{microtype}
  \UseMicrotypeSet[protrusion]{basicmath} % disable protrusion for tt fonts
}{}
\makeatletter
\@ifundefined{KOMAClassName}{% if non-KOMA class
  \IfFileExists{parskip.sty}{%
    \usepackage{parskip}
  }{% else
    \setlength{\parindent}{0pt}
    \setlength{\parskip}{6pt plus 2pt minus 1pt}}
}{% if KOMA class
  \KOMAoptions{parskip=half}}
\makeatother
\usepackage{graphicx}
\makeatletter
\newsavebox\pandoc@box
\newcommand*\pandocbounded[1]{% scales image to fit in text height/width
  \sbox\pandoc@box{#1}%
  \Gscale@div\@tempa{\textheight}{\dimexpr\ht\pandoc@box+\dp\pandoc@box\relax}%
  \Gscale@div\@tempb{\linewidth}{\wd\pandoc@box}%
  \ifdim\@tempb\p@<\@tempa\p@\let\@tempa\@tempb\fi% select the smaller of both
  \ifdim\@tempa\p@<\p@\scalebox{\@tempa}{\usebox\pandoc@box}%
  \else\usebox{\pandoc@box}%
  \fi%
}
% Set default figure placement to htbp
\def\fps@figure{htbp}
\makeatother
\setlength{\emergencystretch}{3em} % prevent overfull lines
\providecommand{\tightlist}{%
  \setlength{\itemsep}{0pt}\setlength{\parskip}{0pt}}
\usepackage{bookmark}
\IfFileExists{xurl.sty}{\usepackage{xurl}}{} % add URL line breaks if available
\urlstyle{same}
\hypersetup{
  pdftitle={CelticInvasivePlantsdb Manual},
  pdfauthor={Claudia González-Toral, Luz Madrazo-Frías, Aránzazu Estrada Fernández, Ricardo López-Alonso, Mauro Sanna, Candela Cuesta, Eduardo Cires \& Juan Viruel},
  hidelinks,
  pdfcreator={LaTeX via pandoc}}

\title{CelticInvasivePlantsdb Manual}
\author{Claudia González-Toral, Luz Madrazo-Frías, Aránzazu Estrada
Fernández, Ricardo López-Alonso, Mauro Sanna, Candela Cuesta, Eduardo
Cires \& Juan Viruel}
\date{09/12/2025}

\begin{document}
\maketitle

{
\setcounter{tocdepth}{2}
\tableofcontents
}
\section{Introduction}\label{introduction}

*This package allows to explore and select the data from the Celtic
Invasive Plants database
(\url{https://doi.org/10.5281/zenodo.17871899}). It allows to select
specific areas or taxa, to generate automatic reports by area or taxon
and obtain distribution and richness maps.

\subsection{Features of the Celtic Invasive Plants
database}\label{features-of-the-celtic-invasive-plants-database}

\begin{itemize}
\tightlist
\item
  Presents occurrence of Alien Invasive Species (AIS) at a 10×10 km UTM
  grid resolution
\item
  Each occurrence is annotated with taxonomic, administrative and
  biogeography data
\item
  The origin status at country and Celtic Fringe scope has been
  annotated
\item
  The AIS checklist specifies the original name of the taxa in each
  local AIS checklist
\end{itemize}

\section{Instalation of
CelticInvasivePlantsdb}\label{instalation-of-celticinvasiveplantsdb}

*This package was upload to a GitHub repository, therefore the users can
use bothe devtools and remotes package to download and install the
library.

\subsection{Instalation with devtools}\label{instalation-with-devtools}

\emph{\#First install and load the devtools library
}install.packages(``devtools'') \emph{library(devtools) }\#Download and
install the package using the GitHub repository
\emph{devtools::install\_github(``Cgt93/CelticInvasivePlantsdb'') }\#Now
load the library *library(CelticInvasivePlantsdb)

\subsection{Instalation with
devtools}\label{instalation-with-devtools-1}

\emph{\#First install and load the remotes library
}install.packages(``remotes'') *library(remotes)

\emph{\#Now we download and instal the package using the GitHub
repository }remotes::install\_github(``Cgt93/CelticInvasivePlantsdb'')
\emph{\#Now load the library }library(CelticInvasivePlantsdb)

\emph{Use Examples}

library(CelticInvasivePlantsdb) \# Código de ejemplo

\emph{References} - Biodiversity in Ireland, 2025. Invasive species of
Ireland. Biodiversity in Ireland. Maps.
\url{https://maps.biodiversityireland.ie/Species} (accessed 12.6.25). -
Boyle, B., Hopkins, N., Lu, Z., Raygoza Garay, J.A., Mozzherin, D.,
Rees, T., Matasci, N., Narro, M.L., Piel, W.H., Mckay, S.J., Lowry, S.,
Freeland, C., Peet, R.K., Enquist, B.J., 2013. The taxonomic name
resolution service: An online tool for automated standardization of
plant names. BMC Bioinformatics 14, 1--15.
\url{https://doi.org/10.1186/1471-2105-14-16} - Boyle, B.L., Matasci,
N., Mozzherin, D., Rees, T., Barbosa, G.C., Kumar Sajja, R., Enquist,
B.J., 2021. Taxonomic Name Resolution Service, version 5.1 . Botanical
Information and Ecology Network. \url{https://tnrs.biendata.org/}
(accessed 12.6.25). - González-Toral, C., Madrazo-Frías, L., Estrada
Fernández, A., López-Alonso, R., Sanna, M., Cuesta, C., Cires, E. \&
Viruel, J. (202) Celtic Invasive Plants database. Version December 2025.
Zenodo.org \url{https://doi.org/10.5281/zenodo.17871899} - Department
for Environment Food \& Rural Affairs and Animal and Plant Health, 2024.
Invasive non-native (alien) plant species: rules in England and Wales.
Gov.UK.
\url{https://www.gov.uk/guidance/invasive-non-native-alien-plant-species-rules-in-england-and-wales\#list-of-invasive-plant-species}
(accessed 12.6.25). - Dudley, N. (Ed.), 2008. Guidelines for Applying
Protected Area Management Categories. IUCN Publications Services, Gland,
Switzerland. - European Commission, 2016. Commission Implementing
Regulation (EU) 2016/1141 of 13 July 2016 adopting a list of invasive
alien species of Union concern pursuant to Regulation (EU) No 1143/2014
of the European Parliament and of the Council. OJ L 189, 14.7.2016,
pp.~4--8. C/2016/4295.
\url{http://data.europa.eu/eli/reg_impl/2016/1141/oj} - European
Commission, 2017. Commission Implementing Regulation (EU) 2017/1263 of
12 July 2017 updating the list of invasive alien species of Union
concern established by Implementing Regulation (EU) 2016/1141 pursuant
to Regulation (EU) No 1143/2014 of the European Parliament and of the
Council. OJ L 182, 13.7.2017, pp.~37--39. C/2017/4755.
\url{http://data.europa.eu/eli/reg_impl/2017/1263/oj} - European
Commission, 2019. Commission Implementing Regulation (EU) 2019/1262 of
25 July 2019 amending Implementing Regulation (EU) 2016/1141 to update
the list of invasive alien species of Union concern. OJ L 199,
26.7.2019, pp.~1--4. C/2019/5360.
\url{http://data.europa.eu/eli/reg_impl/2019/1262/oj} - European
Commission, 2022. Commission Implementing Regulation (EU) 2022/1203 of
12 July 2022 amending Implementing Regulation (EU) 2016/1141 to update
the list of invasive alien species of Union concern. OJ L 186,
13.7.2022, pp.~10--13. C/2022/4773.
\url{http://data.europa.eu/eli/reg_impl/2022/1203/oj} - European
Commission, 2025. Commission Implementing Regulation (EU) 2025/1422 of
17 July 2025 amending Implementing Regulation (EU) 2016/1141 to update
the list of invasive alien species of Union concern. OJ L, 2025/1422,
18.7.2025. C/2025/4769.
\url{http://data.europa.eu/eli/reg_impl/2025/1422/oj} - European Union,
1992. Council Directive 92/43/EEC of 21 May 1992 on the conservation of
natural habitats and of wild fauna and flora. OJ L 206, 22.7.1992,
pp.~7--50. \url{http://data.europa.eu/eli/dir/1992/43/oj} - European
Union, 2009. Directive 2009/147/EC of the European Parliament and of the
Council of 30 November 2009 on the conservation of wild birds (Codified
version). OJ L 20, 26.1.2010, pp.~7--25.
\url{http://data.europa.eu/eli/dir/2009/147/oj} - European Environment
Agency (EEA), 2025a. Emerald Network data (vector) - the Pan-European
network of protected sites version 2024
\url{https://doi.org/10.2909/135a0bb6-c611-4c2c-823d-a564be119ad8} -
European Environment Agency (EEA), 2024. Nationally designated areas for
public access (vector data) - May 2024
\url{https://doi.org/10.2909/616ef48f-7196-4e30-b201-6c97808fa68a} -
European Environment Agency (EEA), 2025b. Natura 2000 (tabular) -
version end 2023
\url{https://www.eea.europa.eu/en/datahub/datahubitem-view/6fc8ad2d-195d-40f4-bdec-576e7d1268e4}
- Fernández Prieto, J.A., Amigo, J., Bueno, A., Herrera, M.,
Rodríguez-Guitián, M.A., Loidi, J., 2020. Notas sobre el Catálogo de
comunidades de plantas vasculares de los territorios iberoatlánticos
(I). Nat. Cantab. 8, 17--37.
\url{https://www.indurot.uniovi.es/actividades/publicaciones/caturalia-cantabricae/volumen-8}
- GB non-native species secretariat 2025. Non-Native Species Secretariat
(NNSS) Species of Special Concern. Non-Native Species Secretariat
(NNSS).
\url{https://www.nonnativespecies.org/legislation/species-of-special-concern\#List-plants}
(accessed 12.3.25). - Govaerts, R. (ed.), 2023. WCVP: World Checklist of
Vascular Plants, Version 12. Royal Botanic Gardens, Kew.
\url{https://sftp.kew.org/pub/data-repositories/WCVP/} (accessed
12.7.25). - Govaerts, R., Nic Lughadha, E., Black, N., Turner, R.,
Paton, A., 2021. The World Checklist of Vascular Plants, a continuously
updated resource for exploring global plant diversity. Sci. Data 8,
1--10. \url{https://doi.org/10.1038/s41597-021-00997-6} - Instituto
Geográfico Nacional, 2024. España. Regiones biogeográficas.Instituto
Geográfico Nacional. Centro de descargas. URL
\url{https://centrodedescargas.cnig.es/CentroDescargas/busquedaRedirigida.do?ruta=PUBLICACION_CNIG_DATOS_VARIOS/aneTematico/Espana_Regiones-biogeograficas_2024_mapa_19246_spa.zip}
(accessed 11.1.25). - Inventaire National du Patrimoine Naturel (INPN),
2025. ERéférentiel taxonomique des espèces des territoires français.
Référentiel taxonomique (Tax Ref) version 18.
\url{https://www.patrinat.fr/fr/page-temporaire-de-telechargement-des-referentiels-de-donnees-lies-linpn-7353}
(accessed 12.6.25). - Minister for Arts Heritage and the Gaeltacht,
2011. European Communities (Birds and Natural Habitats) Regulations
2011. Wt. (B28719). 500. 9/11.
\url{https://www.irishstatutebook.ie/eli/2011/si/477} - Minister for
Housing Local Government and Heritage, 2024. Statutory Instruments.
European Union (Invasive Alien Species) Regulations 2024. Iris Oifigiúil
(IEAD-1) 30. 7/24. Propylon.
\url{https://www.irishstatutebook.ie/eli/2024/si/374/made/en/print} -
Ministère de la Transition Écologique et Solidaire, 2018. Arrêté du 14
février 2018 relatif à la prévention de l'introduction et de la
propagation des espèces végétales exotiques envahissantes sur le
territoire métropolitain. JORF n°0044 du 22 février 2018.NOR :
TREL1704132A.
\url{https://www.legifrance.gouv.fr/loda/id/JORFTEXT000036629837/} -
Ministère de la Transition Écologique et Solidaire, 2020. Arrêté du 10
mars 2020 portant mise à jour de la liste des espèces animales et
végétales exotiques envahissantes sur le territoire métropolitain. JORF
n°0118 du 14 mai 2020, Texte n° 7. NOR : TREL1924265A.
\url{https://www.legifrance.gouv.fr/jorf/id/JORFTEXT000041875937} -
Ministeriet for Fødevarer Landbrug og Fiskeri, 2018. Bekendtgørelse om
forebyggelse og håndtering af introduktion og spredning af invasive
ikkehjemmehørende arter på EU-listen og om en national liste med
handelsforbud m.v. over for invasive arter. BEK nr 1285 af 12/11/2018.
\url{https://www.retsinformation.dk/eli/lta/2018/1285} - Ministeriet for
Grøn Trepart, 2025. De invasive arter. De invasive artslister. Arter.
\url{https://sgavmst.dk/arter/artsforvaltning/invasive-arter/de-invasive-arter}
(accessed 12.5.25). - Ministerio para la Transición Ecológica, 2019.
Real Decreto 216/2019, de 29 de marzo, por el que se aprueba la lista de
especies exóticas invasoras preocupantes para la región ultraperiférica
de las islas Canarias y por el que se modifica el Real Decreto 630/2013,
de 2 de agosto, por el que se regula el Catálogo español de especies
exóticas invasoras. «BOE» núm. 77, de 30/03/2019. BOE-A-2019-4675.
\url{https://www.boe.es/eli/es/rd/2019/03/29/216/con} - Ministerio para
la Transición Ecológica y el Reto Demográfico, 2020. Orden
TED/1126/2020, de 20 de noviembre, por la que se modifica el Anexo del
Real Decreto 139/2011, de 4 de febrero, para el desarrollo del Listado
de Especies Silvestres en Régimen de Protección Especial y del Catálogo
Español de Especies Amenazadas, y el Anexo del Real Decreto 630/2013, de
2 de agosto, por el que se regula el Catálogo Español de Especies
Exóticas Invasoras.BOE-A-2020-15296. «BOE» núm. 314, de 1 de diciembre
de 2020, páginas 108167 a 108171 (5 págs.).
\url{https://www.boe.es/eli/es/o/2020/11/20/ted1126} - Ministerio para
la Transición Ecológica y el Reto Demográfico, 2023a. Orden
TED/339/2023, de 30 de marzo, por la que se modifica el anexo del Real
Decreto 139/2011, de 4 de febrero, para el desarrollo del Listado de
Especies Silvestres en Régimen de Protección Especial y del Catálogo
Español de Especies Amenazadas, y el anexo del Real Decreto 630/2013, de
2 de agosto, por el que se regula el Catálogo Español de Especies
Exóticas Invasoras.«BOE» núm. 83, de 7 de abril de 2023, páginas 50910 a
50915 (6 págs.). BOE-A-2023-8751.
\url{https://www.boe.es/eli/es/o/2023/03/30/ted339} - Ministerio para la
Transición Ecológica y el Reto Demográfico, 2023b. Catálogo Español de
Especies Exóticas Invasoras. MITECO. URL
\url{https://www.miteco.gob.es/es/biodiversidad/temas/conservacion-de-especies/especies-exoticas-invasoras/ce-eei-catalogo.aspx}
(accessed 6.11.23). - National Biodiversity Data Centre Of the Republic
of Ireland, 2023. Discrete vascular plant surveys. Data.Gov.IE.
\url{https://data.gov.ie/dataset/discrete-vascular-plant-surveys}
(accessed 3.10.23). - Presidência do Conselho de Ministros Ambiente e
Transição Energética, 2019. Assegura a execução, na ordem jurídica
nacional, do Regulamento (UE) n.o 1143/2014, estabelecendo o regime
jurídico aplicável ao controlo, à detenção, à introdução na natureza e
ao repovoamento de espécies exóticas da flora e da fauna. Diário da
República n.º 130/2019, Série I de 2019-07-10. Decreto-Lei n.º 92/2019.
\url{https://diariodarepublica.pt/dr/legislacao-consolidada/decreto-lei/2019-124568069}
- Rees, T., 2014. Taxamatch, an Algorithm for Near (`Fuzzy') Matching of
Scientific Names in Taxonomic Databases. PLoS One 9, e107510.
\url{https://doi.org/10.1371/journal.pone.0107510} - Rivas-Martínez, S.,
Penas, A., Díaz; T. E., 2001. Biogeographic map of Europe. Cartographic
Service University of León, León. - Royal Botanic Gardens Kew, 2025.
Plants of the World Online (POWO). Facilitated by the Royal Botanic
Gardens, Kew. \url{http://www.plantsoftheworldonline.org/} (accessed
4.1.25). - Royal Horticultural Society (RHS), 2025. Invasive plants
covered by legislation. RHS.org
\url{https://www.rhs.org.uk/prevention-protection/invasive-non-native-plants}
(accessed 12.4.25). - Stolton, S., Shadie, P., Dudley, N., 2013. IUCN
WCPA Best Practice Guidance on Recognising Protected Areas and Assigning
Management Categories and Governance Types. Best Practice Protected Area
Guidelines Series 21.
\url{https://portals.iucn.org/library/sites/library/files/documents/pag-021.pdf}
- The Angiosperm Phylogeny Group, 2016. An update of the Angiosperm
Phylogeny Group classification for the orders and families of flowering
plants: APG IV. Bot. J. Linn. Soc. 181, 399--436.
\url{https://doi.org/10.1111/boj.12385} - The World Flora Online
Consortium, Elliott, A., Hyam, R., Ulate, W., 2023. World Flora Online
Plant List June 2023. Version 2023-06. Zenodo.org.
\url{https://zenodo.org/records/8079052} (accessed 12.7.25).
\url{https://doi.org/10.5281/zenodo.8079052} - Thomas, S., 2011. Natural
England Commissioned Report NECR053: Horizon-scanning for invasive
non-native plants in Great Britain (NECR053). Natural England.
\url{https://publications.naturalengland.org.uk/publication/40015}

\end{document}
